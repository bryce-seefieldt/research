\documentclass{article}
\usepackage{cite}
\usepackage{hyperref}

\title{Literature Review: The Impact of AI-Generated Content on Copyright Law and Artists’ Rights}
\author{Your Name}
\date{October 2024}

\begin{document}

\maketitle

\tableofcontents
\newpage

\section{Introduction to the Literature Review}
The purpose of this literature review is to provide a comprehensive overview of the current state of research on three key themes: the rise of AI-generated content in visual arts, legal definitions and debates surrounding copyright infringement related to AI, and the moral and economic rights of visual artists. This review synthesizes existing literature to identify patterns, gaps, and unresolved issues that justify the need for further research.

\section{The Rise of AI-Generated Content in Visual Arts}
AI-generated content has seen a significant rise in the visual arts, transforming creative processes and challenging traditional notions of authorship. Recent advancements in AI technologies, such as Generative Adversarial Networks (GANs), text-to-image (T2I) models like DALL-E, and other generative algorithms, have enabled artists and technologists to create high-quality visual content through automated means. According to Liu et al. (2024), the NTIRE 2024 Challenge demonstrated the growing sophistication of AI-generated images and videos, emphasizing the need for standardization in quality assessment \cite{liu2024ntire}.

Moreover, the relationship between art and technology, as discussed by Indrajaya and Rizky (2018), is evolving towards a more conceptual approach. The authors argue that modern AI tools are shifting the focus from the process of creation to the idea behind the artwork, similar to the impact of conceptual art in the 20th century \cite{indrajaya2018techne}.

\section{Legal Definitions and Debates Around Copyright Infringement}
The rise of AI-generated content has raised significant legal questions, particularly regarding authorship and originality. Çebi, Reisoğlu, and Goktas (2023) explore the legal ambiguities in copyright protection for AI-generated works, noting that the concept of originality—central to copyright law—becomes blurred with the involvement of AI \cite{cebi2023influence}. They highlight key cases and policy proposals within the European Union (EU) and the World Intellectual Property Organization (WIPO) that aim to address these challenges.

Peukert and Windisch (2024) examine how digital technologies, including AI, are reshaping copyright enforcement mechanisms. They argue that current frameworks need adaptation to address issues such as algorithmic licensing and copyright exceptions for training datasets, which are critical for the functioning of AI models \cite{peukert2024economics}.

\section{Moral and Economic Rights of Visual Artists}
AI-generated content also poses challenges to the moral and economic rights of visual artists. Peters and Cartwright (2023) investigate the potential of non-fungible tokens (NFTs) to empower artists by establishing verifiable ownership and royalty mechanisms. However, they emphasize that the lack of standardized regulations creates legal ambiguities around IP rights for digital assets like NFTs \cite{peters2023nfts}.

Indrajaya and Rizky (2018) provide a philosophical critique of the instrumentalist approach, where technology commodifies art, thereby undermining the role of the artist. They suggest that AI-generated art can lead to appropriation of artistic styles without proper attribution, raising concerns about authenticity and moral rights \cite{indrajaya2018techne}.

\section{Gaps and Unresolved Issues}
This literature review has identified several key gaps and unresolved issues:
\begin{itemize}
    \item \textbf{Lack of Standardized Legal Definitions:} Conflicting legal definitions of authorship and originality exist across jurisdictions, complicating the international enforcement of copyright protections for AI-generated works \cite{cebi2023influence}.
    \item \textbf{Uncertainty Around Ownership Models:} The rise of NFTs introduces new ownership models that have not yet been fully integrated into existing legal frameworks, raising concerns about IP rights and attribution \cite{peters2023nfts}.
\end{itemize}

\section{Conclusion}
AI-generated content is revolutionizing the visual arts, but it also presents significant legal and ethical challenges. The need for standardized legal definitions and frameworks to address authorship, ownership, and moral rights is evident. Future research should focus on developing cohesive policies and regulations to protect the interests of both artists and creators of AI-generated content.

\bibliographystyle{unsrt}
\bibliography{references}

\end{document}
